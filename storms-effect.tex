% Options for packages loaded elsewhere
\PassOptionsToPackage{unicode}{hyperref}
\PassOptionsToPackage{hyphens}{url}
%
\documentclass[
]{article}
\usepackage{lmodern}
\usepackage{amssymb,amsmath}
\usepackage{ifxetex,ifluatex}
\ifnum 0\ifxetex 1\fi\ifluatex 1\fi=0 % if pdftex
  \usepackage[T1]{fontenc}
  \usepackage[utf8]{inputenc}
  \usepackage{textcomp} % provide euro and other symbols
\else % if luatex or xetex
  \usepackage{unicode-math}
  \defaultfontfeatures{Scale=MatchLowercase}
  \defaultfontfeatures[\rmfamily]{Ligatures=TeX,Scale=1}
\fi
% Use upquote if available, for straight quotes in verbatim environments
\IfFileExists{upquote.sty}{\usepackage{upquote}}{}
\IfFileExists{microtype.sty}{% use microtype if available
  \usepackage[]{microtype}
  \UseMicrotypeSet[protrusion]{basicmath} % disable protrusion for tt fonts
}{}
\makeatletter
\@ifundefined{KOMAClassName}{% if non-KOMA class
  \IfFileExists{parskip.sty}{%
    \usepackage{parskip}
  }{% else
    \setlength{\parindent}{0pt}
    \setlength{\parskip}{6pt plus 2pt minus 1pt}}
}{% if KOMA class
  \KOMAoptions{parskip=half}}
\makeatother
\usepackage{xcolor}
\IfFileExists{xurl.sty}{\usepackage{xurl}}{} % add URL line breaks if available
\IfFileExists{bookmark.sty}{\usepackage{bookmark}}{\usepackage{hyperref}}
\hypersetup{
  pdftitle={storms\_and\_severe\_weather},
  pdfauthor={Ahmed Hassan},
  hidelinks,
  pdfcreator={LaTeX via pandoc}}
\urlstyle{same} % disable monospaced font for URLs
\usepackage[margin=1in]{geometry}
\usepackage{color}
\usepackage{fancyvrb}
\newcommand{\VerbBar}{|}
\newcommand{\VERB}{\Verb[commandchars=\\\{\}]}
\DefineVerbatimEnvironment{Highlighting}{Verbatim}{commandchars=\\\{\}}
% Add ',fontsize=\small' for more characters per line
\usepackage{framed}
\definecolor{shadecolor}{RGB}{248,248,248}
\newenvironment{Shaded}{\begin{snugshade}}{\end{snugshade}}
\newcommand{\AlertTok}[1]{\textcolor[rgb]{0.94,0.16,0.16}{#1}}
\newcommand{\AnnotationTok}[1]{\textcolor[rgb]{0.56,0.35,0.01}{\textbf{\textit{#1}}}}
\newcommand{\AttributeTok}[1]{\textcolor[rgb]{0.77,0.63,0.00}{#1}}
\newcommand{\BaseNTok}[1]{\textcolor[rgb]{0.00,0.00,0.81}{#1}}
\newcommand{\BuiltInTok}[1]{#1}
\newcommand{\CharTok}[1]{\textcolor[rgb]{0.31,0.60,0.02}{#1}}
\newcommand{\CommentTok}[1]{\textcolor[rgb]{0.56,0.35,0.01}{\textit{#1}}}
\newcommand{\CommentVarTok}[1]{\textcolor[rgb]{0.56,0.35,0.01}{\textbf{\textit{#1}}}}
\newcommand{\ConstantTok}[1]{\textcolor[rgb]{0.00,0.00,0.00}{#1}}
\newcommand{\ControlFlowTok}[1]{\textcolor[rgb]{0.13,0.29,0.53}{\textbf{#1}}}
\newcommand{\DataTypeTok}[1]{\textcolor[rgb]{0.13,0.29,0.53}{#1}}
\newcommand{\DecValTok}[1]{\textcolor[rgb]{0.00,0.00,0.81}{#1}}
\newcommand{\DocumentationTok}[1]{\textcolor[rgb]{0.56,0.35,0.01}{\textbf{\textit{#1}}}}
\newcommand{\ErrorTok}[1]{\textcolor[rgb]{0.64,0.00,0.00}{\textbf{#1}}}
\newcommand{\ExtensionTok}[1]{#1}
\newcommand{\FloatTok}[1]{\textcolor[rgb]{0.00,0.00,0.81}{#1}}
\newcommand{\FunctionTok}[1]{\textcolor[rgb]{0.00,0.00,0.00}{#1}}
\newcommand{\ImportTok}[1]{#1}
\newcommand{\InformationTok}[1]{\textcolor[rgb]{0.56,0.35,0.01}{\textbf{\textit{#1}}}}
\newcommand{\KeywordTok}[1]{\textcolor[rgb]{0.13,0.29,0.53}{\textbf{#1}}}
\newcommand{\NormalTok}[1]{#1}
\newcommand{\OperatorTok}[1]{\textcolor[rgb]{0.81,0.36,0.00}{\textbf{#1}}}
\newcommand{\OtherTok}[1]{\textcolor[rgb]{0.56,0.35,0.01}{#1}}
\newcommand{\PreprocessorTok}[1]{\textcolor[rgb]{0.56,0.35,0.01}{\textit{#1}}}
\newcommand{\RegionMarkerTok}[1]{#1}
\newcommand{\SpecialCharTok}[1]{\textcolor[rgb]{0.00,0.00,0.00}{#1}}
\newcommand{\SpecialStringTok}[1]{\textcolor[rgb]{0.31,0.60,0.02}{#1}}
\newcommand{\StringTok}[1]{\textcolor[rgb]{0.31,0.60,0.02}{#1}}
\newcommand{\VariableTok}[1]{\textcolor[rgb]{0.00,0.00,0.00}{#1}}
\newcommand{\VerbatimStringTok}[1]{\textcolor[rgb]{0.31,0.60,0.02}{#1}}
\newcommand{\WarningTok}[1]{\textcolor[rgb]{0.56,0.35,0.01}{\textbf{\textit{#1}}}}
\usepackage{graphicx,grffile}
\makeatletter
\def\maxwidth{\ifdim\Gin@nat@width>\linewidth\linewidth\else\Gin@nat@width\fi}
\def\maxheight{\ifdim\Gin@nat@height>\textheight\textheight\else\Gin@nat@height\fi}
\makeatother
% Scale images if necessary, so that they will not overflow the page
% margins by default, and it is still possible to overwrite the defaults
% using explicit options in \includegraphics[width, height, ...]{}
\setkeys{Gin}{width=\maxwidth,height=\maxheight,keepaspectratio}
% Set default figure placement to htbp
\makeatletter
\def\fps@figure{htbp}
\makeatother
\setlength{\emergencystretch}{3em} % prevent overfull lines
\providecommand{\tightlist}{%
  \setlength{\itemsep}{0pt}\setlength{\parskip}{0pt}}
\setcounter{secnumdepth}{-\maxdimen} % remove section numbering

\title{storms\_and\_severe\_weather}
\author{Ahmed Hassan}
\date{12/19/2020}

\begin{document}
\maketitle

\hypertarget{impact-of-storms-and-other-severe-weather-events-on-both-public-health-and-economic}{%
\section{impact of Storms and other severe weather events on both public
health and
economic}\label{impact-of-storms-and-other-severe-weather-events-on-both-public-health-and-economic}}

\hypertarget{synopsis}{%
\subsection{Synopsis}\label{synopsis}}

The basic goal of this report is to answer some basic questions about
Storms and severe weather events such as,\\
1- which types of events are most harmful with respect to population
health?\\
2- which types of events have the greatest economic consequences?\\
to answer this questions we obtained the U.S. National Oceanic and
Atmospheric Administration's (NOAA) storm database. This database tracks
characteristics of major storms and weather events in the United States,
including when and where they occur, as well as estimates of any
fatalities, injuries, and property damage.

\hypertarget{loading-and-processing-the-raw-data.}{%
\subsection{Loading and Processing the Raw
Data.}\label{loading-and-processing-the-raw-data.}}

From the
\href{https://d396qusza40orc.cloudfront.net/repdata\%2Fdata\%2FStormData.csv.bz2}{Storm
Data} we can explore and answer our mysterious questions.\\
There is also some documentation of the database available. Here you
will find how some of the variables are constructed/defined.\\
* National Weather Service
\href{https://d396qusza40orc.cloudfront.net/repdata\%2Fpeer2_doc\%2Fpd01016005curr.pdf}{Storm
Data Documentation}\\
* National Climatic Data Center Storm Events
\href{https://d396qusza40orc.cloudfront.net/repdata\%2Fpeer2_doc\%2FNCDC\%20Storm\%20Events-FAQ\%20Page.pdf}{FAQ}

\hypertarget{load-the-relevant-libraries-unzip-and-reading-the-data}{%
\subsection{load the relevant libraries, unzip and reading the
data}\label{load-the-relevant-libraries-unzip-and-reading-the-data}}

\begin{Shaded}
\begin{Highlighting}[]
\CommentTok{# load the relevant libraries}
\KeywordTok{library}\NormalTok{(ggplot2)}
\KeywordTok{library}\NormalTok{(dplyr)}
\end{Highlighting}
\end{Shaded}

\begin{verbatim}
## 
## Attaching package: 'dplyr'
\end{verbatim}

\begin{verbatim}
## The following objects are masked from 'package:stats':
## 
##     filter, lag
\end{verbatim}

\begin{verbatim}
## The following objects are masked from 'package:base':
## 
##     intersect, setdiff, setequal, union
\end{verbatim}

\begin{Shaded}
\begin{Highlighting}[]
\KeywordTok{library}\NormalTok{(plyr)}
\end{Highlighting}
\end{Shaded}

\begin{verbatim}
## ------------------------------------------------------------------------------
\end{verbatim}

\begin{verbatim}
## You have loaded plyr after dplyr - this is likely to cause problems.
## If you need functions from both plyr and dplyr, please load plyr first, then dplyr:
## library(plyr); library(dplyr)
\end{verbatim}

\begin{verbatim}
## ------------------------------------------------------------------------------
\end{verbatim}

\begin{verbatim}
## 
## Attaching package: 'plyr'
\end{verbatim}

\begin{verbatim}
## The following objects are masked from 'package:dplyr':
## 
##     arrange, count, desc, failwith, id, mutate, rename, summarise,
##     summarize
\end{verbatim}

\begin{Shaded}
\begin{Highlighting}[]
\KeywordTok{library}\NormalTok{(lubridate)}
\end{Highlighting}
\end{Shaded}

\begin{verbatim}
## 
## Attaching package: 'lubridate'
\end{verbatim}

\begin{verbatim}
## The following objects are masked from 'package:base':
## 
##     date, intersect, setdiff, union
\end{verbatim}

\begin{Shaded}
\begin{Highlighting}[]
\NormalTok{stormData <-}\StringTok{ }\KeywordTok{read.csv}\NormalTok{(}\StringTok{"repdata_data_StormData.csv.bz2"}\NormalTok{)}
\end{Highlighting}
\end{Shaded}

\hypertarget{explore-the-data}{%
\subsection{Explore the data}\label{explore-the-data}}

\begin{Shaded}
\begin{Highlighting}[]
\KeywordTok{str}\NormalTok{(stormData)}
\end{Highlighting}
\end{Shaded}

\begin{verbatim}
## 'data.frame':    902297 obs. of  37 variables:
##  $ STATE__   : num  1 1 1 1 1 1 1 1 1 1 ...
##  $ BGN_DATE  : chr  "4/18/1950 0:00:00" "4/18/1950 0:00:00" "2/20/1951 0:00:00" "6/8/1951 0:00:00" ...
##  $ BGN_TIME  : chr  "0130" "0145" "1600" "0900" ...
##  $ TIME_ZONE : chr  "CST" "CST" "CST" "CST" ...
##  $ COUNTY    : num  97 3 57 89 43 77 9 123 125 57 ...
##  $ COUNTYNAME: chr  "MOBILE" "BALDWIN" "FAYETTE" "MADISON" ...
##  $ STATE     : chr  "AL" "AL" "AL" "AL" ...
##  $ EVTYPE    : chr  "TORNADO" "TORNADO" "TORNADO" "TORNADO" ...
##  $ BGN_RANGE : num  0 0 0 0 0 0 0 0 0 0 ...
##  $ BGN_AZI   : chr  "" "" "" "" ...
##  $ BGN_LOCATI: chr  "" "" "" "" ...
##  $ END_DATE  : chr  "" "" "" "" ...
##  $ END_TIME  : chr  "" "" "" "" ...
##  $ COUNTY_END: num  0 0 0 0 0 0 0 0 0 0 ...
##  $ COUNTYENDN: logi  NA NA NA NA NA NA ...
##  $ END_RANGE : num  0 0 0 0 0 0 0 0 0 0 ...
##  $ END_AZI   : chr  "" "" "" "" ...
##  $ END_LOCATI: chr  "" "" "" "" ...
##  $ LENGTH    : num  14 2 0.1 0 0 1.5 1.5 0 3.3 2.3 ...
##  $ WIDTH     : num  100 150 123 100 150 177 33 33 100 100 ...
##  $ F         : int  3 2 2 2 2 2 2 1 3 3 ...
##  $ MAG       : num  0 0 0 0 0 0 0 0 0 0 ...
##  $ FATALITIES: num  0 0 0 0 0 0 0 0 1 0 ...
##  $ INJURIES  : num  15 0 2 2 2 6 1 0 14 0 ...
##  $ PROPDMG   : num  25 2.5 25 2.5 2.5 2.5 2.5 2.5 25 25 ...
##  $ PROPDMGEXP: chr  "K" "K" "K" "K" ...
##  $ CROPDMG   : num  0 0 0 0 0 0 0 0 0 0 ...
##  $ CROPDMGEXP: chr  "" "" "" "" ...
##  $ WFO       : chr  "" "" "" "" ...
##  $ STATEOFFIC: chr  "" "" "" "" ...
##  $ ZONENAMES : chr  "" "" "" "" ...
##  $ LATITUDE  : num  3040 3042 3340 3458 3412 ...
##  $ LONGITUDE : num  8812 8755 8742 8626 8642 ...
##  $ LATITUDE_E: num  3051 0 0 0 0 ...
##  $ LONGITUDE_: num  8806 0 0 0 0 ...
##  $ REMARKS   : chr  "" "" "" "" ...
##  $ REFNUM    : num  1 2 3 4 5 6 7 8 9 10 ...
\end{verbatim}

\hypertarget{create-a-subset-for-relevant-data.}{%
\subsection{Create a subset for relevant
data.}\label{create-a-subset-for-relevant-data.}}

The relevant elements for the analysis are the date (BGN\_DATE), the
type of event (EVTYPE), the health impact counter (FATALITIES and
INJURIES), the monetary impact on crops and goods (PROPDMG and CROPDMG)
as well as their corresponding exponents (PROPDMGEXP and CROPDMGEXP).

\begin{Shaded}
\begin{Highlighting}[]
\CommentTok{## Create a subset for relevant data.}
\NormalTok{relevantStormData <-}\StringTok{ }\KeywordTok{select}\NormalTok{(stormData,  }\KeywordTok{c}\NormalTok{(}\StringTok{"EVTYPE"}\NormalTok{, }\StringTok{"FATALITIES"}\NormalTok{, }\StringTok{"INJURIES"}\NormalTok{, }\StringTok{"PROPDMG"}\NormalTok{, }\StringTok{"PROPDMGEXP"}\NormalTok{, }\StringTok{"CROPDMG"}\NormalTok{, }\StringTok{"CROPDMGEXP"}\NormalTok{))}
\CommentTok{# Explore the relevant data}
\KeywordTok{str}\NormalTok{(relevantStormData)}
\end{Highlighting}
\end{Shaded}

\begin{verbatim}
## 'data.frame':    902297 obs. of  7 variables:
##  $ EVTYPE    : chr  "TORNADO" "TORNADO" "TORNADO" "TORNADO" ...
##  $ FATALITIES: num  0 0 0 0 0 0 0 0 1 0 ...
##  $ INJURIES  : num  15 0 2 2 2 6 1 0 14 0 ...
##  $ PROPDMG   : num  25 2.5 25 2.5 2.5 2.5 2.5 2.5 25 25 ...
##  $ PROPDMGEXP: chr  "K" "K" "K" "K" ...
##  $ CROPDMG   : num  0 0 0 0 0 0 0 0 0 0 ...
##  $ CROPDMGEXP: chr  "" "" "" "" ...
\end{verbatim}

\hypertarget{checking-for-missing-values.}{%
\subsection{Checking for missing
values.}\label{checking-for-missing-values.}}

looping through the columns (with sapply) to get the number of NAs,
because Missing values are a common problem with environmental data and
so we check to se what proportion of the observations are missing

\begin{Shaded}
\begin{Highlighting}[]
\KeywordTok{sapply}\NormalTok{(relevantStormData, }\ControlFlowTok{function}\NormalTok{(x) }\KeywordTok{sum}\NormalTok{(}\KeywordTok{is.na}\NormalTok{(x)))}
\end{Highlighting}
\end{Shaded}

\begin{verbatim}
##     EVTYPE FATALITIES   INJURIES    PROPDMG PROPDMGEXP    CROPDMG CROPDMGEXP 
##          0          0          0          0          0          0          0
\end{verbatim}

\hypertarget{adjust-property-damagepropdmgexp-and-crop-damage-cropdmgexp}{%
\subsection{adjust Property Damage(PROPDMGEXP) and crop damage
(CROPDMGEXP)}\label{adjust-property-damagepropdmgexp-and-crop-damage-cropdmgexp}}

transform the value of the variable PROPDMGEXP \& CROPDMGEXP from
character such as ``K'' to number to be able to calculate the total
property damage \& total crop damage.\\
so We're going to convert the exponents into corresponding factors:
``'', ``?'', ``+'', ``-'': 1 ``0'': 1 ``1'': 10 ``2'': 100 ``3'': 1.000
``4'': 10.000 ``5'': 100.000 ``6'': 1.000.000 ``7'': 10.000.000 ``8'':
100.000.000 ``9'': 1.000.000.000 ``H'': 100 ``K'': 1.000 ``M'':
1.000.000 *``B'': 1.000.000.000

\begin{Shaded}
\begin{Highlighting}[]
\NormalTok{x1 <-}\StringTok{ }\KeywordTok{unique}\NormalTok{(relevantStormData}\OperatorTok{$}\NormalTok{PROPDMGEXP)}
\NormalTok{x2 <-}\StringTok{ }\KeywordTok{c}\NormalTok{(}\DecValTok{10}\OperatorTok{^}\DecValTok{3}\NormalTok{, }\DecValTok{10}\OperatorTok{^}\DecValTok{6}\NormalTok{, }\DecValTok{1}\NormalTok{, }\DecValTok{10}\OperatorTok{^}\DecValTok{9}\NormalTok{, }\DecValTok{10}\OperatorTok{^}\DecValTok{6}\NormalTok{, }\DecValTok{0}\NormalTok{,}\DecValTok{1}\NormalTok{,}\DecValTok{10}\OperatorTok{^}\DecValTok{5}\NormalTok{, }\DecValTok{10}\OperatorTok{^}\DecValTok{6}\NormalTok{, }\DecValTok{0}\NormalTok{, }\DecValTok{10}\OperatorTok{^}\DecValTok{4}\NormalTok{, }\DecValTok{10}\OperatorTok{^}\DecValTok{2}\NormalTok{, }\DecValTok{10}\OperatorTok{^}\DecValTok{3}\NormalTok{, }\DecValTok{10}\OperatorTok{^}\DecValTok{2}\NormalTok{, }\DecValTok{10}\OperatorTok{^}\DecValTok{7}\NormalTok{, }\DecValTok{10}\OperatorTok{^}\DecValTok{2}\NormalTok{, }\DecValTok{0}\NormalTok{, }\DecValTok{10}\NormalTok{, }\DecValTok{10}\OperatorTok{^}\DecValTok{8}\NormalTok{)}
\NormalTok{y1 <-}\StringTok{ }\KeywordTok{unique}\NormalTok{(relevantStormData}\OperatorTok{$}\NormalTok{CROPDMGEXP)}
\NormalTok{y2 <-}\StringTok{ }\KeywordTok{c}\NormalTok{(}\DecValTok{1}\NormalTok{,}\DecValTok{10}\OperatorTok{^}\DecValTok{6}\NormalTok{, }\DecValTok{10}\OperatorTok{^}\DecValTok{3}\NormalTok{, }\DecValTok{10}\OperatorTok{^}\DecValTok{6}\NormalTok{, }\DecValTok{10}\OperatorTok{^}\DecValTok{9}\NormalTok{, }\DecValTok{0}\NormalTok{, }\DecValTok{1}\NormalTok{, }\DecValTok{10}\OperatorTok{^}\DecValTok{3}\NormalTok{, }\DecValTok{10}\OperatorTok{^}\DecValTok{2}\NormalTok{)}
\NormalTok{relevantStormData}\OperatorTok{$}\NormalTok{PROPDMGEXP <-}\StringTok{ }\KeywordTok{mapvalues}\NormalTok{(relevantStormData}\OperatorTok{$}\NormalTok{PROPDMGEXP, }\DataTypeTok{from =}\NormalTok{ x1, }\DataTypeTok{to =}\NormalTok{ x2)}
\NormalTok{relevantStormData}\OperatorTok{$}\NormalTok{PROPDMGEXP <-}\StringTok{ }\KeywordTok{as.numeric}\NormalTok{(}\KeywordTok{as.character}\NormalTok{(relevantStormData}\OperatorTok{$}\NormalTok{PROPDMGEXP))}
\NormalTok{relevantStormData}\OperatorTok{$}\NormalTok{CROPDMGEXP <-}\StringTok{ }\KeywordTok{mapvalues}\NormalTok{(relevantStormData}\OperatorTok{$}\NormalTok{CROPDMGEXP, }\DataTypeTok{from =}\NormalTok{ y1, }\DataTypeTok{to =}\NormalTok{ y2)}
\NormalTok{relevantStormData}\OperatorTok{$}\NormalTok{CROPDMGEXP <-}\StringTok{ }\KeywordTok{as.numeric}\NormalTok{(}\KeywordTok{as.character}\NormalTok{(relevantStormData}\OperatorTok{$}\NormalTok{CROPDMGEXP))}
\end{Highlighting}
\end{Shaded}

calculate the total number of Property Damage(PROPDMGEXP) and crop
damage (CROPDMGEXP)

\begin{Shaded}
\begin{Highlighting}[]
\NormalTok{relevantStormData}\OperatorTok{$}\NormalTok{PROPDMGTOTAL <-}\StringTok{ }\NormalTok{(relevantStormData}\OperatorTok{$}\NormalTok{PROPDMG }\OperatorTok{*}\StringTok{ }\NormalTok{relevantStormData}\OperatorTok{$}\NormalTok{PROPDMGEXP)}\OperatorTok{/}\DecValTok{1000000000}
\NormalTok{relevantStormData}\OperatorTok{$}\NormalTok{CROPDMGTOTAL <-}\StringTok{ }\NormalTok{(relevantStormData}\OperatorTok{$}\NormalTok{CROPDMG }\OperatorTok{*}\StringTok{ }\NormalTok{relevantStormData}\OperatorTok{$}\NormalTok{CROPDMGEXP)}\OperatorTok{/}\DecValTok{1000000000}
\end{Highlighting}
\end{Shaded}

\hypertarget{processing-the-data-for-analysis.}{%
\subsubsection{Processing the data for
analysis.}\label{processing-the-data-for-analysis.}}

\hypertarget{events-for-public-health-variables.}{%
\subsection{Events for public health
variables.}\label{events-for-public-health-variables.}}

\hypertarget{fatalities}{%
\section{Fatalities}\label{fatalities}}

\begin{Shaded}
\begin{Highlighting}[]
\NormalTok{aggFatalites <-}\StringTok{ }\KeywordTok{aggregate}\NormalTok{(FATALITIES }\OperatorTok{~}\StringTok{ }\NormalTok{EVTYPE, }\DataTypeTok{data =}\NormalTok{ relevantStormData,  }\DataTypeTok{FUN=}\StringTok{"sum"}\NormalTok{)}
\KeywordTok{dim}\NormalTok{(aggFatalites) }
\end{Highlighting}
\end{Shaded}

\begin{verbatim}
## [1] 985   2
\end{verbatim}

\hypertarget{screen-the-top-10-weather-events-of-fatalities}{%
\section{Screen the top 10 Weather events of
fatalities}\label{screen-the-top-10-weather-events-of-fatalities}}

\begin{Shaded}
\begin{Highlighting}[]
\NormalTok{top10Fatalities <-}\StringTok{ }\NormalTok{aggFatalites[}\KeywordTok{order}\NormalTok{(}\OperatorTok{-}\NormalTok{aggFatalites}\OperatorTok{$}\NormalTok{FATALITIES), ][}\DecValTok{1}\OperatorTok{:}\DecValTok{10}\NormalTok{, ]}
\NormalTok{top10Fatalities}
\end{Highlighting}
\end{Shaded}

\begin{verbatim}
##             EVTYPE FATALITIES
## 834        TORNADO       5633
## 130 EXCESSIVE HEAT       1903
## 153    FLASH FLOOD        978
## 275           HEAT        937
## 464      LIGHTNING        816
## 856      TSTM WIND        504
## 170          FLOOD        470
## 585    RIP CURRENT        368
## 359      HIGH WIND        248
## 19       AVALANCHE        224
\end{verbatim}

Plot the histogram

\begin{Shaded}
\begin{Highlighting}[]
\NormalTok{fatalitiesPlot <-}\StringTok{ }\KeywordTok{ggplot}\NormalTok{(}\DataTypeTok{data =}\NormalTok{ top10Fatalities, }\KeywordTok{aes}\NormalTok{(}\DataTypeTok{x =} \KeywordTok{reorder}\NormalTok{(EVTYPE, FATALITIES), }\DataTypeTok{y =}\NormalTok{ FATALITIES, }\DataTypeTok{color =}\NormalTok{ EVTYPE)) }\OperatorTok{+}\StringTok{ }\KeywordTok{geom_bar}\NormalTok{(}\DataTypeTok{stat=}\StringTok{"identity"}\NormalTok{,}\DataTypeTok{fill=}\StringTok{"white"}\NormalTok{) }\OperatorTok{+}\StringTok{ }\KeywordTok{xlab}\NormalTok{(}\StringTok{"Event Type"}\NormalTok{) }\OperatorTok{+}\StringTok{  }\KeywordTok{ylab}\NormalTok{(}\StringTok{"Total number of fatalities"}\NormalTok{) }\OperatorTok{+}\StringTok{  }\KeywordTok{ggtitle}\NormalTok{(}\StringTok{"10 Fatalities Highest Events"}\NormalTok{) }

\NormalTok{fatalitiesPlot }\OperatorTok{+}\StringTok{ }\KeywordTok{coord_flip}\NormalTok{()}
\end{Highlighting}
\end{Shaded}

\includegraphics{storm-effect_files/figure-latex/unnamed-chunk-9-1.pdf}
\# Injuries

\begin{Shaded}
\begin{Highlighting}[]
\NormalTok{aggInjuries <-}\StringTok{ }\KeywordTok{aggregate}\NormalTok{(INJURIES }\OperatorTok{~}\StringTok{ }\NormalTok{EVTYPE, }\DataTypeTok{data =}\NormalTok{ relevantStormData,  }\DataTypeTok{FUN=}\StringTok{"sum"}\NormalTok{)}
\KeywordTok{dim}\NormalTok{(aggInjuries) }
\end{Highlighting}
\end{Shaded}

\begin{verbatim}
## [1] 985   2
\end{verbatim}

\hypertarget{screen-the-top-10-weather-events-of-fatalities-1}{%
\section{Screen the top 10 Weather events of
fatalities}\label{screen-the-top-10-weather-events-of-fatalities-1}}

\begin{Shaded}
\begin{Highlighting}[]
\NormalTok{top10Injuries <-}\StringTok{ }\NormalTok{aggInjuries[}\KeywordTok{order}\NormalTok{(}\OperatorTok{-}\NormalTok{aggInjuries}\OperatorTok{$}\NormalTok{INJURIES), ][}\DecValTok{1}\OperatorTok{:}\DecValTok{10}\NormalTok{, ]}
\NormalTok{top10Injuries}
\end{Highlighting}
\end{Shaded}

\begin{verbatim}
##                EVTYPE INJURIES
## 834           TORNADO    91346
## 856         TSTM WIND     6957
## 170             FLOOD     6789
## 130    EXCESSIVE HEAT     6525
## 464         LIGHTNING     5230
## 275              HEAT     2100
## 427         ICE STORM     1975
## 153       FLASH FLOOD     1777
## 760 THUNDERSTORM WIND     1488
## 244              HAIL     1361
\end{verbatim}

\hypertarget{plot-the-histogram}{%
\section{Plot the histogram}\label{plot-the-histogram}}

\begin{Shaded}
\begin{Highlighting}[]
\NormalTok{injuriesPlot <-}\StringTok{ }\KeywordTok{ggplot}\NormalTok{(}\DataTypeTok{data =}\NormalTok{ top10Injuries, }\KeywordTok{aes}\NormalTok{(}\DataTypeTok{x =} \KeywordTok{reorder}\NormalTok{(EVTYPE, INJURIES), }\DataTypeTok{y =}\NormalTok{ INJURIES, }\DataTypeTok{color=}\NormalTok{EVTYPE)) }\OperatorTok{+}\StringTok{ }\KeywordTok{geom_bar}\NormalTok{(}\DataTypeTok{stat=}\StringTok{"identity"}\NormalTok{,}\DataTypeTok{fill=}\StringTok{"white"}\NormalTok{) }\OperatorTok{+}\StringTok{ }\KeywordTok{xlab}\NormalTok{(}\StringTok{"Event Type"}\NormalTok{) }\OperatorTok{+}\StringTok{  }\KeywordTok{ylab}\NormalTok{(}\StringTok{"Total number of injuries"}\NormalTok{) }\OperatorTok{+}\StringTok{  }\KeywordTok{ggtitle}\NormalTok{(}\StringTok{"10 Injuries Highest Events"}\NormalTok{) }

\NormalTok{injuriesPlot }\OperatorTok{+}\StringTok{ }\KeywordTok{coord_flip}\NormalTok{()}
\end{Highlighting}
\end{Shaded}

\includegraphics{storm-effect_files/figure-latex/unnamed-chunk-12-1.pdf}
\#\# Property Damage

\begin{Shaded}
\begin{Highlighting}[]
\NormalTok{aggPdamage <-}\StringTok{ }\KeywordTok{aggregate}\NormalTok{(PROPDMGTOTAL }\OperatorTok{~}\StringTok{ }\NormalTok{EVTYPE, }\DataTypeTok{data =}\NormalTok{ relevantStormData,  }\DataTypeTok{FUN=}\StringTok{"sum"}\NormalTok{)}
\KeywordTok{dim}\NormalTok{(aggPdamage) }
\end{Highlighting}
\end{Shaded}

\begin{verbatim}
## [1] 985   2
\end{verbatim}

Screen the top 10 Property damage Events

\begin{Shaded}
\begin{Highlighting}[]
\NormalTok{top10Pdamage <-}\StringTok{ }\NormalTok{aggPdamage[}\KeywordTok{order}\NormalTok{(}\OperatorTok{-}\NormalTok{aggPdamage}\OperatorTok{$}\NormalTok{PROPDMGTOTAL), ][}\DecValTok{1}\OperatorTok{:}\DecValTok{10}\NormalTok{, ]}
\NormalTok{top10Pdamage}
\end{Highlighting}
\end{Shaded}

\begin{verbatim}
##                EVTYPE PROPDMGTOTAL
## 170             FLOOD   144.657710
## 411 HURRICANE/TYPHOON    69.305840
## 834           TORNADO    56.947381
## 670       STORM SURGE    43.323536
## 153       FLASH FLOOD    16.822674
## 244              HAIL    15.735268
## 402         HURRICANE    11.868319
## 848    TROPICAL STORM     7.703891
## 972      WINTER STORM     6.688497
## 359         HIGH WIND     5.270046
\end{verbatim}

Plot the histogram

\begin{Shaded}
\begin{Highlighting}[]
\NormalTok{pdamagePlot <-}\StringTok{ }\KeywordTok{ggplot}\NormalTok{(}\DataTypeTok{data =}\NormalTok{ top10Pdamage, }\KeywordTok{aes}\NormalTok{(}\DataTypeTok{x =} \KeywordTok{reorder}\NormalTok{(EVTYPE, PROPDMGTOTAL), }\DataTypeTok{y =}\NormalTok{ PROPDMGTOTAL, }\DataTypeTok{color =}\NormalTok{ EVTYPE)) }\OperatorTok{+}\StringTok{ }\KeywordTok{geom_bar}\NormalTok{(}\DataTypeTok{stat =} \StringTok{"identity"}\NormalTok{,}\DataTypeTok{fill=}\StringTok{"white"}\NormalTok{) }\OperatorTok{+}\StringTok{ }\KeywordTok{xlab}\NormalTok{(}\StringTok{"Event Type"}\NormalTok{) }\OperatorTok{+}\StringTok{  }\KeywordTok{ylab}\NormalTok{(}\StringTok{"Total damage in dollars"}\NormalTok{) }\OperatorTok{+}\StringTok{  }\KeywordTok{ggtitle}\NormalTok{(}\StringTok{"10 Highest Property Damages Events"}\NormalTok{) }

\NormalTok{pdamagePlot }\OperatorTok{+}\StringTok{ }\KeywordTok{coord_flip}\NormalTok{()}
\end{Highlighting}
\end{Shaded}

\includegraphics{storm-effect_files/figure-latex/unnamed-chunk-15-1.pdf}
\#\# Crop Damage

\begin{Shaded}
\begin{Highlighting}[]
\NormalTok{aggCdamage <-}\StringTok{ }\KeywordTok{aggregate}\NormalTok{(CROPDMGTOTAL }\OperatorTok{~}\StringTok{ }\NormalTok{EVTYPE, }\DataTypeTok{data =}\NormalTok{ relevantStormData,  }\DataTypeTok{FUN=}\StringTok{"sum"}\NormalTok{)}
\KeywordTok{dim}\NormalTok{(aggCdamage) }
\end{Highlighting}
\end{Shaded}

\begin{verbatim}
## [1] 985   2
\end{verbatim}

\hypertarget{screen-the-top-10-property-damage-events}{%
\section{Screen the top 10 Property damage
Events}\label{screen-the-top-10-property-damage-events}}

\begin{Shaded}
\begin{Highlighting}[]
\NormalTok{top10Cdamage <-}\StringTok{ }\NormalTok{aggCdamage[}\KeywordTok{order}\NormalTok{(}\OperatorTok{-}\NormalTok{aggCdamage}\OperatorTok{$}\NormalTok{CROPDMGTOTAL), ][}\DecValTok{1}\OperatorTok{:}\DecValTok{10}\NormalTok{, ]}
\NormalTok{top10Pdamage}
\end{Highlighting}
\end{Shaded}

\begin{verbatim}
##                EVTYPE PROPDMGTOTAL
## 170             FLOOD   144.657710
## 411 HURRICANE/TYPHOON    69.305840
## 834           TORNADO    56.947381
## 670       STORM SURGE    43.323536
## 153       FLASH FLOOD    16.822674
## 244              HAIL    15.735268
## 402         HURRICANE    11.868319
## 848    TROPICAL STORM     7.703891
## 972      WINTER STORM     6.688497
## 359         HIGH WIND     5.270046
\end{verbatim}

plot the histogram

\begin{Shaded}
\begin{Highlighting}[]
\NormalTok{cdamagePlot <-}\StringTok{ }\KeywordTok{ggplot}\NormalTok{(}\DataTypeTok{data =}\NormalTok{ top10Cdamage, }\KeywordTok{aes}\NormalTok{(}\DataTypeTok{x =} \KeywordTok{reorder}\NormalTok{(EVTYPE, CROPDMGTOTAL), }\DataTypeTok{y =}\NormalTok{ CROPDMGTOTAL, }\DataTypeTok{color=}\NormalTok{EVTYPE)) }\OperatorTok{+}\StringTok{ }\KeywordTok{geom_bar}\NormalTok{(}\DataTypeTok{stat=}\StringTok{"identity"}\NormalTok{,}\DataTypeTok{fill=}\StringTok{"white"}\NormalTok{) }\OperatorTok{+}\StringTok{ }\KeywordTok{xlab}\NormalTok{(}\StringTok{"Event Type"}\NormalTok{) }\OperatorTok{+}\StringTok{  }\KeywordTok{ylab}\NormalTok{(}\StringTok{"Total crop in dollars"}\NormalTok{) }\OperatorTok{+}\StringTok{  }\KeywordTok{ggtitle}\NormalTok{(}\StringTok{"10 Highest Crop Damages Events"}\NormalTok{) }

\NormalTok{cdamagePlot }\OperatorTok{+}\StringTok{ }\KeywordTok{coord_flip}\NormalTok{()}
\end{Highlighting}
\end{Shaded}

\includegraphics{storm-effect_files/figure-latex/unnamed-chunk-18-1.pdf}

\hypertarget{results}{%
\subsubsection{Results}\label{results}}

\hypertarget{question-1}{%
\subsection{Question 1}\label{question-1}}

\hypertarget{the-histogram-shows-that-tornados-are-the-most-harmful-weather-events-for-peoples-health.}{%
\section{The histogram shows that Tornados are the most harmful weather
events for people's
health.}\label{the-histogram-shows-that-tornados-are-the-most-harmful-weather-events-for-peoples-health.}}

\hypertarget{question-2}{%
\subsection{Question 2}\label{question-2}}

\hypertarget{the-histogram-shows-that-floods-cause-the-biggest-property-damages.-the-histogram-shows-that-drought-cause-the-biggest-crop-damages.}{%
\section{The histogram shows that Floods cause the biggest Property
damages. The histogram shows that DROUGHT cause the biggest Crop
damages.}\label{the-histogram-shows-that-floods-cause-the-biggest-property-damages.-the-histogram-shows-that-drought-cause-the-biggest-crop-damages.}}

\end{document}
